%%%%%%%%%%%%%%%%%%%%%%%%%%%%%%%%%%%%%%%%%
% Lachaise Assignment
% LaTeX Template
% Version 1.0 (26/6/2018)
%
% This template originates from:
% http://www.LaTeXTemplates.com
%
% Authors:
% Marion Lachaise & François Févotte
% Vel (vel@LaTeXTemplates.com)
%
% License:
% CC BY-NC-SA 3.0 (http://creativecommons.org/licenses/by-nc-sa/3.0/)
%
%%%%%%%%%%%%%%%%%%%%%%%%%%%%%%%%%%%%%%%%%

%----------------------------------------------------------------------------------------
%	PACKAGES AND OTHER DOCUMENT CONFIGURATIONS
%----------------------------------------------------------------------------------------
\documentclass{article}

%%%%%%%%%%%%%%%%%%%%%%%%%%%%%%%%%%%%%%%%%
% Lachaise Assignment
% Structure Specification File
% Version 1.0 (26/6/2018)
%
% This template originates from:
% http://www.LaTeXTemplates.com
%
% Authors:
% Marion Lachaise & François Févotte
% Vel (vel@LaTeXTemplates.com)
%
% License:
% CC BY-NC-SA 3.0 (http://creativecommons.org/licenses/by-nc-sa/3.0/)
% 
%%%%%%%%%%%%%%%%%%%%%%%%%%%%%%%%%%%%%%%%%

%----------------------------------------------------------------------------------------
%	PACKAGES AND OTHER DOCUMENT CONFIGURATIONS
%----------------------------------------------------------------------------------------

\usepackage{amsmath,amsfonts,stmaryrd,amssymb} % Math packages

\usepackage{enumerate} % Custom item numbers for enumerations

\usepackage[ruled]{algorithm2e} % Algorithms

\usepackage[framemethod=tikz]{mdframed} % Allows defining custom boxed/framed environments

\usepackage{listings} % File listings, with syntax highlighting
\lstset{
	basicstyle=\ttfamily, % Typeset listings in monospace font
}

%----------------------------------------------------------------------------------------
%	DOCUMENT MARGINS
%----------------------------------------------------------------------------------------

\usepackage{geometry} % Required for adjusting page dimensions and margins

\geometry{
	paper=a4paper, % Paper size, change to letterpaper for US letter size
	top=2.5cm, % Top margin
	bottom=3cm, % Bottom margin
	left=2.5cm, % Left margin
	right=2.5cm, % Right margin
	headheight=14pt, % Header height
	footskip=1.5cm, % Space from the bottom margin to the baseline of the footer
	headsep=1.2cm, % Space from the top margin to the baseline of the header
	%showframe, % Uncomment to show how the type block is set on the page
}

%----------------------------------------------------------------------------------------
%	FONTS
%----------------------------------------------------------------------------------------

\usepackage[utf8]{inputenc} % Required for inputting international characters
\usepackage[T1]{fontenc} % Output font encoding for international characters

\usepackage{XCharter} % Use the XCharter fonts

%----------------------------------------------------------------------------------------
%	COMMAND LINE ENVIRONMENT
%----------------------------------------------------------------------------------------

% Usage:
% \begin{commandline}
%	\begin{verbatim}
%		$ ls
%		
%		Applications	Desktop	...
%	\end{verbatim}
% \end{commandline}

\mdfdefinestyle{commandline}{
	leftmargin=10pt,
	rightmargin=10pt,
	innerleftmargin=15pt,
	middlelinecolor=black!50!white,
	middlelinewidth=2pt,
	frametitlerule=false,
	backgroundcolor=black!5!white,
	frametitle={Command Line},
	frametitlefont={\normalfont\sffamily\color{white}\hspace{-1em}},
	frametitlebackgroundcolor=black!50!white,
	nobreak,
}

% Define a custom environment for command-line snapshots
\newenvironment{commandline}{
	\medskip
	\begin{mdframed}[style=commandline]
}{
	\end{mdframed}
	\medskip
}

%----------------------------------------------------------------------------------------
%	FILE CONTENTS ENVIRONMENT
%----------------------------------------------------------------------------------------

% Usage:
% \begin{file}[optional filename, defaults to "File"]
%	File contents, for example, with a listings environment
% \end{file}

\mdfdefinestyle{file}{
	innertopmargin=1.6\baselineskip,
	innerbottommargin=0.8\baselineskip,
	topline=false, bottomline=false,
	leftline=false, rightline=false,
	leftmargin=2cm,
	rightmargin=2cm,
	singleextra={%
		\draw[fill=black!10!white](P)++(0,-1.2em)rectangle(P-|O);
		\node[anchor=north west]
		at(P-|O){\ttfamily\mdfilename};
		%
		\def\l{3em}
		\draw(O-|P)++(-\l,0)--++(\l,\l)--(P)--(P-|O)--(O)--cycle;
		\draw(O-|P)++(-\l,0)--++(0,\l)--++(\l,0);
	},
	nobreak,
}

% Define a custom environment for file contents
\newenvironment{file}[1][File]{ % Set the default filename to "File"
	\medskip
	\newcommand{\mdfilename}{#1}
	\begin{mdframed}[style=file]
}{
	\end{mdframed}
	\medskip
}

%----------------------------------------------------------------------------------------
%	NUMBERED QUESTIONS ENVIRONMENT
%----------------------------------------------------------------------------------------

% Usage:
% \begin{question}[optional title]
%	Question contents
% \end{question}

\mdfdefinestyle{question}{
	innertopmargin=1.2\baselineskip,
	innerbottommargin=0.8\baselineskip,
	roundcorner=5pt,
	nobreak,
	singleextra={%
		\draw(P-|O)node[xshift=1em,anchor=west,fill=white,draw,rounded corners=5pt]{%
		Question \theQuestion\questionTitle};
	},
}

\newcounter{Question} % Stores the current question number that gets iterated with each new question

% Define a custom environment for numbered questions
\newenvironment{question}[1][\unskip]{
	\bigskip
	\stepcounter{Question}
	\newcommand{\questionTitle}{~#1}
	\begin{mdframed}[style=question]
}{
	\end{mdframed}
	\medskip
}

%----------------------------------------------------------------------------------------
%	WARNING TEXT ENVIRONMENT
%----------------------------------------------------------------------------------------

% Usage:
% \begin{warn}[optional title, defaults to "Warning:"]
%	Contents
% \end{warn}

\mdfdefinestyle{warning}{
	topline=false, bottomline=false,
	leftline=false, rightline=false,
	nobreak,
	singleextra={%
		\draw(P-|O)++(-0.5em,0)node(tmp1){};
		\draw(P-|O)++(0.5em,0)node(tmp2){};
		\fill[black,rotate around={45:(P-|O)}](tmp1)rectangle(tmp2);
		\node at(P-|O){\color{white}\scriptsize\bf !};
		\draw[very thick](P-|O)++(0,-1em)--(O);%--(O-|P);
	}
}

% Define a custom environment for warning text
\newenvironment{warn}[1][Warning:]{ % Set the default warning to "Warning:"
	\medskip
	\begin{mdframed}[style=warning]
		\noindent{\textbf{#1}}
}{
	\end{mdframed}
}

%----------------------------------------------------------------------------------------
%	INFORMATION ENVIRONMENT
%----------------------------------------------------------------------------------------

% Usage:
% \begin{info}[optional title, defaults to "Info:"]
% 	contents
% 	\end{info}

\mdfdefinestyle{info}{%
	topline=false, bottomline=false,
	leftline=false, rightline=false,
	nobreak,
	singleextra={%
		\fill[black](P-|O)circle[radius=0.4em];
		\node at(P-|O){\color{white}\scriptsize\bf i};
		\draw[very thick](P-|O)++(0,-0.8em)--(O);%--(O-|P);
	}
}

% Define a custom environment for information
\newenvironment{info}[1][Info:]{ % Set the default title to "Info:"
	\medskip
	\begin{mdframed}[style=info]
		\noindent{\textbf{#1}}
}{
	\end{mdframed}
}
 % Include the file specifying the document structure and custom commands

%----------------------------------------------------------------------------------------
%	ASSIGNMENT INFORMATION
%----------------------------------------------------------------------------------------

\title{Pricing Models: Assignment \#1} % Title of the assignment

\author{N.Borovykh, E. van Raaij\\  \\ \texttt{Natalia.Borovykh@Rabobank.com} \\ \texttt{Erik.van.Raaij@Rabobank.com}} % Author name and email address

\date{TU Delft --- \today} % University, school and/or department name(s) and a date

%----------------------------------------------------------------------------------------

\begin{document}

\maketitle % Print the title

%----------------------------------------------------------------------------------------
%	INTRODUCTION
%----------------------------------------------------------------------------------------

\section*{Introduction} % Unnumbered section

In this document the first assignment is explained high level. It also provides some clarity on the deliverables that have to be handed in. 
It's expected that you make the assignments in pairs of two using the \textit{TUD01 EuropeanOption.zip} provided to you.

% Math equation/formula
%\begin{equation}
%	I = \int_{a}^{b} f(x) \; \text{d}x.
%\end{equation}


\begin{info} % Information block
The zip file consist of different folders:
\begin{itemize}
  \item Code
  \item Documentation
  \item Input
  \item Output
  \item Solutions
\end{itemize}	
\end{info}
The 'scenario' that generate the results of the assignment should be provided in the Solutions. 
The Code folder contains several files with functions that might be re-used. 
To have readable code these functions are not coded 'inline' but taken out. 
It's advised to scan first the folders and files to see what is available and to get a feel for the structure.

\begin{info} % Information block
Please use Anaconda distribution 2023.07 or higher which can be downloaded from
\begin{itemize}
    \item https://www.anaconda.com/products/individual
\end{itemize}
The assignment will be checked either in Spyder or Visual Studio Code.
\end{info}
\begin{warn}[Deliverable 01:]
  Please add your name and email address to the \textit{Students.csv} file in the Input folder.
\end{warn}

%----------------------------------------------------------------------------------------
%	PROBLEM 1
%----------------------------------------------------------------------------------------
%\newpage
\section{European Option} % Numbered section

In the first assignment European options are priced assuming lognormal dynamics of the underlying. 
First, the analytical distribution is compared to the one simulated with Monte Carlo. 
Second, the price of an analytically calculated option is compared to the price of the same option calculated using Monte Carlo. 
Last, but not least, the prices of puts and calls have to be expressed in volatilities.
%------------------------------------------------

\newpage
\subsection{Lognormal distribution}
\label{sec:LNDistribution}
\begin{info} % Information block
The details of the lognormal underlying are the following:
    \begin{itemize}
      \item spot price  = 443
      \item volatility = 20\%
      \item risk free rate = 6\%
      \item expiry = 2 (year)
    \end{itemize}
\end{info}
Simulate the values of the underlying at expiry with Monte Carlo (MC) and create an empirical distribution using the .hist function of matplotlib.pyplot package. 
As you've to reuse the path generator this should be coded in \textit{paths.py}. 
The lognormal PDF should be added to the \textit{distribution.py}. 
The MC simulation should be run with 2000 (2k) and 20k paths.

\begin{warn}[Deliverable 02:]
  Create one figure with two subplots where the empirical distribution is compared to (the same) analytical distribution. 
  Save the figure to the Output folder. 
  In the first subplot use 2k paths. 
  In the second subplot use 20k paths. 
  The code to generate the figure should be put in \textit{Solution11.py}
\end{warn}

\subsection{Call option prices}
\begin{info} % Information block
See section \ref{sec:LNDistribution}. In addition,
    \begin{itemize}
      \item strike price  = 443
      \item number of paths = [5k, 10k, 50k, 100k, 150k, 200k, 400k, 600k, 800k, 1000k]
    \end{itemize}
\end{info}

Calculate the value of the European option analytically and several times with MC using the given number of paths. 
Therefore add an analytical pricer and a MC pricer to \textit{payoffs.py}. 
The code should be generic such that put prices can be calculated as well. 
The MC pricer should return both the price and its standard error. 
Keep track how long it takes to generate the MC paths and the MC price for each number of paths.

\begin{warn}[Deliverable 03:]
  Create one figure with two subplots. Show in the first plot the analytical price and the MC prices. 
  In addition plot the standard error for each MC run. 
  Show in the second subplot how long each MC run took. 
  Save the figure to the Output folder. 
  The data used to create the figures should be saved in CSV format to the output folder as well. 
  The code to generate the figure and the output data should be put in \textit{Solution12.py}
\end{warn}

\newpage

\subsection{Implied Volatilities}
\begin{info} % Information block
In the Input folder \textit{OptionPrices.csv} is stored. This file contains:
    \begin{itemize}
      \item out of the money (OTM) European put and call prices
      \item spot values
      \item strike prices
      \item expiries
      \item risk free rates
      \item CPs (1 = call, -1 = put)
    \end{itemize}
In addition a threshold is needed, which is set to 0.0001.
\end{info}

Calculate for each option/strike its corresponding implied volatility (IV). 
The functionality to calculate the IV should be added to \textit{bsfunctions.py}. 
The function should also keep track of the number iterations it needs to find the IV given the above threshold.

\begin{warn}[Deliverable 04:]
  Create one figure with two subplots. Show in the first plot both the prices and the volatilities. 
  Plot in the second subplot per option/strike the number of iterations it took to converge. 
  Save the figure to the Output folder. 
  The data used to create the figures should be saved in CSV format to the Output folder as well. 
  The code to generate the figure and the output data should be put in \textit{Solution13.py}
\end{warn}

% Numbered question, with subquestions in an enumerate environment
%\begin{question}
%	Quisque ullamcorper placerat ipsum. Cras nibh. Morbi vel justo vitae lacus tincidunt ultrices. Lorem ipsum dolor sit amet, consectetuer adipiscing elit.
%
%	% Subquestions numbered with letters
%	\begin{enumerate}[(a)]
%		\item Do this.
%		\item Do that.
%		\item Do something else.
%	\end{enumerate}
%\end{question}
	
%------------------------------------------------


%\begin{center}
%	\begin{minipage}{0.5\linewidth} % Adjust the minipage width to accomodate for the length of algorithm lines
%		\begin{algorithm}[H]
%			\KwIn{$(a, b)$, two floating-point numbers}  % Algorithm inputs
%			\KwResult{$(c, d)$, such that $a+b = c + d$} % Algorithm outputs/results
%			\medskip
%			\If{$\vert b\vert > \vert a\vert$}{
%				exchange $a$ and $b$ \;
%			}
%			$c \leftarrow a + b$ \;
%			$z \leftarrow c - a$ \;
%			$d \leftarrow b - z$ \;
%			{\bf return} $(c,d)$ \;
%			\caption{\texttt{FastTwoSum}} % Algorithm name
%			\label{alg:fastTwoSum}   % optional label to refer to
%		\end{algorithm}
%	\end{minipage}
%\end{center}


% Numbered question, with an optional title
%\begin{question}[\itshape (with optional title)
%Proin hendrerit sem nec tempor sollicitudin.
%\end{question}


%----------------------------------------------------------------------------------------
%	PROBLEM 2
%----------------------------------------------------------------------------------------


% File contents
%\begin{file}[hello.py]
%\begin{lstlisting}[language=Python]
%#! /usr/bin/python
%
%import sys
%sys.stdout.write("Hello World!\n")
%\end{lstlisting}
%\end{file}

% Command-line "screenshot"

%\begin{commandline}
%	\begin{verbatim}
%		$ chmod +x hello.py
%		$ ./hello.py
%
%		Hello World!
%	\end{verbatim}
%\end{commandline}



\end{document}
